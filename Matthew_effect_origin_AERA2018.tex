\documentclass[english,floatsintext,jou]{apa6}

\usepackage{amssymb,amsmath}
\usepackage{ifxetex,ifluatex}
\usepackage{fixltx2e} % provides \textsubscript
\ifnum 0\ifxetex 1\fi\ifluatex 1\fi=0 % if pdftex
  \usepackage[T1]{fontenc}
  \usepackage[utf8]{inputenc}
\else % if luatex or xelatex
  \ifxetex
    \usepackage{mathspec}
    \usepackage{xltxtra,xunicode}
  \else
    \usepackage{fontspec}
  \fi
  \defaultfontfeatures{Mapping=tex-text,Scale=MatchLowercase}
  \newcommand{\euro}{€}
\fi
% use upquote if available, for straight quotes in verbatim environments
\IfFileExists{upquote.sty}{\usepackage{upquote}}{}
% use microtype if available
\IfFileExists{microtype.sty}{\usepackage{microtype}}{}
\usepackage{color}
\usepackage{fancyvrb}
\newcommand{\VerbBar}{|}
\newcommand{\VERB}{\Verb[commandchars=\\\{\}]}
\DefineVerbatimEnvironment{Highlighting}{Verbatim}{commandchars=\\\{\}}
% Add ',fontsize=\small' for more characters per line
\usepackage{framed}
\definecolor{shadecolor}{RGB}{248,248,248}
\newenvironment{Shaded}{\begin{snugshade}}{\end{snugshade}}
\newcommand{\KeywordTok}[1]{\textcolor[rgb]{0.13,0.29,0.53}{\textbf{{#1}}}}
\newcommand{\DataTypeTok}[1]{\textcolor[rgb]{0.13,0.29,0.53}{{#1}}}
\newcommand{\DecValTok}[1]{\textcolor[rgb]{0.00,0.00,0.81}{{#1}}}
\newcommand{\BaseNTok}[1]{\textcolor[rgb]{0.00,0.00,0.81}{{#1}}}
\newcommand{\FloatTok}[1]{\textcolor[rgb]{0.00,0.00,0.81}{{#1}}}
\newcommand{\ConstantTok}[1]{\textcolor[rgb]{0.00,0.00,0.00}{{#1}}}
\newcommand{\CharTok}[1]{\textcolor[rgb]{0.31,0.60,0.02}{{#1}}}
\newcommand{\SpecialCharTok}[1]{\textcolor[rgb]{0.00,0.00,0.00}{{#1}}}
\newcommand{\StringTok}[1]{\textcolor[rgb]{0.31,0.60,0.02}{{#1}}}
\newcommand{\VerbatimStringTok}[1]{\textcolor[rgb]{0.31,0.60,0.02}{{#1}}}
\newcommand{\SpecialStringTok}[1]{\textcolor[rgb]{0.31,0.60,0.02}{{#1}}}
\newcommand{\ImportTok}[1]{{#1}}
\newcommand{\CommentTok}[1]{\textcolor[rgb]{0.56,0.35,0.01}{\textit{{#1}}}}
\newcommand{\DocumentationTok}[1]{\textcolor[rgb]{0.56,0.35,0.01}{\textbf{\textit{{#1}}}}}
\newcommand{\AnnotationTok}[1]{\textcolor[rgb]{0.56,0.35,0.01}{\textbf{\textit{{#1}}}}}
\newcommand{\CommentVarTok}[1]{\textcolor[rgb]{0.56,0.35,0.01}{\textbf{\textit{{#1}}}}}
\newcommand{\OtherTok}[1]{\textcolor[rgb]{0.56,0.35,0.01}{{#1}}}
\newcommand{\FunctionTok}[1]{\textcolor[rgb]{0.00,0.00,0.00}{{#1}}}
\newcommand{\VariableTok}[1]{\textcolor[rgb]{0.00,0.00,0.00}{{#1}}}
\newcommand{\ControlFlowTok}[1]{\textcolor[rgb]{0.13,0.29,0.53}{\textbf{{#1}}}}
\newcommand{\OperatorTok}[1]{\textcolor[rgb]{0.81,0.36,0.00}{\textbf{{#1}}}}
\newcommand{\BuiltInTok}[1]{{#1}}
\newcommand{\ExtensionTok}[1]{{#1}}
\newcommand{\PreprocessorTok}[1]{\textcolor[rgb]{0.56,0.35,0.01}{\textit{{#1}}}}
\newcommand{\AttributeTok}[1]{\textcolor[rgb]{0.77,0.63,0.00}{{#1}}}
\newcommand{\RegionMarkerTok}[1]{{#1}}
\newcommand{\InformationTok}[1]{\textcolor[rgb]{0.56,0.35,0.01}{\textbf{\textit{{#1}}}}}
\newcommand{\WarningTok}[1]{\textcolor[rgb]{0.56,0.35,0.01}{\textbf{\textit{{#1}}}}}
\newcommand{\AlertTok}[1]{\textcolor[rgb]{0.94,0.16,0.16}{{#1}}}
\newcommand{\ErrorTok}[1]{\textcolor[rgb]{0.64,0.00,0.00}{\textbf{{#1}}}}
\newcommand{\NormalTok}[1]{{#1}}

% Table formatting
\usepackage{longtable, booktabs}
\usepackage{lscape}
% \usepackage[counterclockwise]{rotating}   % Landscape page setup for large tables
\usepackage{multirow}		% Table styling
\usepackage{tabularx}		% Control Column width
\usepackage[flushleft]{threeparttable}	% Allows for three part tables with a specified notes section
\usepackage{threeparttablex}            % Lets threeparttable work with longtable

% Create new environments so endfloat can handle them
% \newenvironment{ltable}
%   {\begin{landscape}\begin{center}\begin{threeparttable}}
%   {\end{threeparttable}\end{center}\end{landscape}}

\newenvironment{lltable}
  {\begin{landscape}\begin{center}\begin{ThreePartTable}}
  {\end{ThreePartTable}\end{center}\end{landscape}}




% The following enables adjusting longtable caption width to table width
% Solution found at http://golatex.de/longtable-mit-caption-so-breit-wie-die-tabelle-t15767.html
\makeatletter
\newcommand\LastLTentrywidth{1em}
\newlength\longtablewidth
\setlength{\longtablewidth}{1in}
\newcommand\getlongtablewidth{%
 \begingroup
  \ifcsname LT@\roman{LT@tables}\endcsname
  \global\longtablewidth=0pt
  \renewcommand\LT@entry[2]{\global\advance\longtablewidth by ##2\relax\gdef\LastLTentrywidth{##2}}%
  \@nameuse{LT@\roman{LT@tables}}%
  \fi
\endgroup}


  \usepackage{graphicx}
  \makeatletter
  \def\maxwidth{\ifdim\Gin@nat@width>\linewidth\linewidth\else\Gin@nat@width\fi}
  \def\maxheight{\ifdim\Gin@nat@height>\textheight\textheight\else\Gin@nat@height\fi}
  \makeatother
  % Scale images if necessary, so that they will not overflow the page
  % margins by default, and it is still possible to overwrite the defaults
  % using explicit options in \includegraphics[width, height, ...]{}
  \setkeys{Gin}{width=\maxwidth,height=\maxheight,keepaspectratio}
\ifxetex
  \usepackage[setpagesize=false, % page size defined by xetex
              unicode=false, % unicode breaks when used with xetex
              xetex]{hyperref}
\else
  \usepackage[unicode=true]{hyperref}
\fi
\hypersetup{breaklinks=true,
            pdfauthor={},
            pdftitle={On the Origin of the Matthew Effect: Insights from a Quantitative Theoretical Model},
            colorlinks=true,
            citecolor=blue,
            urlcolor=blue,
            linkcolor=black,
            pdfborder={0 0 0}}
\urlstyle{same}  % don't use monospace font for urls

\setlength{\parindent}{0pt}
%\setlength{\parskip}{0pt plus 0pt minus 0pt}

\setlength{\emergencystretch}{3em}  % prevent overfull lines

\ifxetex
  \usepackage{polyglossia}
  \setmainlanguage{}
\else
  \usepackage[english]{babel}
\fi

% Manuscript styling
\captionsetup{font=singlespacing,justification=justified}
\usepackage{csquotes}
\usepackage{upgreek}



\usepackage{tikz} % Variable definition to generate author note

% fix for \tightlist problem in pandoc 1.14
\providecommand{\tightlist}{%
  \setlength{\itemsep}{0pt}\setlength{\parskip}{0pt}}

% Essential manuscript parts
  \title{On the Origin of the Matthew Effect: Insights from a Quantitative
Theoretical Model}

  \shorttitle{Origin}


  \author{Matthew T. McBee\textsuperscript{1}, Matthew C. Makel\textsuperscript{2}, \& Natasha L. Godkin\textsuperscript{1}}

  % \def\affdep{{"", "", ""}}%
  % \def\affcity{{"", "", ""}}%

  \affiliation{
    \vspace{0.5cm}
          \textsuperscript{1} East Tennessee State University\\
          \textsuperscript{2} Duke University Talent Identification Program  }

  \authornote{
    Correspondence concerning this article should be addressed to Matthew T.
    McBee, . E-mail:
    \href{mailto:mcbeem@etsu.edu}{\nolinkurl{mcbeem@etsu.edu}}
  }


  \abstract{A quantitative theoretical model of academic achievement, based on the
concept of the Zone of Proximal Development (ZPD) is used to explore the
origin and development of achievement discrepancies via the phenomenon
known as the Matthew effect. This paper applies the model to explore the
potential impact of various types of educational interventions and their
anticipated effects on academic achievement as well as excellence and
achievement gaps. Interventions tend to aggravate both type of gaps
unless they are paired with countervailing curricular insufficiency that
serves to limit the growth of the highest achieving students.}
  \keywords{gifted, achievement gap, excellence gap, zone of proximal development \\

    \indent Word count: X
  }





\usepackage{amsthm}
\newtheorem{theorem}{Theorem}[section]
\newtheorem{lemma}{Lemma}[section]
\theoremstyle{definition}
\newtheorem{definition}{Definition}[section]
\newtheorem{corollary}{Corollary}[section]
\newtheorem{proposition}{Proposition}[section]
\theoremstyle{definition}
\newtheorem{example}{Example}[section]
\theoremstyle{definition}
\newtheorem{exercise}{Exercise}[section]
\theoremstyle{remark}
\newtheorem*{remark}{Remark}
\newtheorem*{solution}{Solution}
\begin{document}

\maketitle

\setcounter{secnumdepth}{0}



\enquote{For whoever has will be given more, and they will have an
abundance. Whoever does not have, even what they have will be taken from
them.} Matthew 25:29. NIV

The Biblical passage quote above, more colloquially stated, \enquote{the
rich get richer and the poor get poorer} originally referred to the
growth of a person's faith. Merton (1968), relying on Zuckerman (1965)
interviews with Nobel laureates, argued that a modernized form of what
he dubbed \enquote{The Matthew Effect} applied to scientists. As one
Nobel interviewee stated, \enquote{The world tends to give the credit to
already famous people} (Merton, p.~607). Broadly, those who had an
initial advantage or accomplishment, reaped not just rewards, but
outsized rewards, which lead to subsequent outsized opportunity and
advantage which further widened the gap between those who had from those
who had not. Merton described the Nobel winners as sometimes being
embarrassed beneficiaries of this phenomenon.

Since Merton, others have found cumulative advantage Matthew Effects in
numerous domains including education (for review, see Ceci and Papierno
(2005)). For example, students who read well and have strong
vocabularies tend read more than their peers, thus further developing
their reading skills whereas children with weaker vocabularies read more
slowly and less, thus widening the gap in reading between the groups
(Ceci \& Papierno, 2005). Ceci and Papierno also note that Matthew
Effects are observed when both gifted and nongifted students are
provided with an educational intervention (Borkowski \& Peck, 1986); an
already existing gap between the groups grew larger after the
intervention.

There is not universal acceptance of Matthew Effects. For example,
researchers have found that a diverging performance that had a
fan-spread pattern was not evident. However, they found that performance
differences in the initial ability was maintained across grades
(Protopapas, Sideridis, Mouzaki, \& Simons, 2011). The general
alternative hypothesis to the Matthew Effect is called the compensatory
trajectory of development. Children who initially start with lower
abilities in reading, accelerate their abilities over time and catch up
to those who had higher initial starting abilities (Huang, Moon, \&
Boren, 2014). Regardless, to our knowledge, no one suggests that Matthew
Effects are inevitable. Rather, that they are both common and pervasive
(Ceci \& Papierno, 2005).

\par

Matthew Effects are particularly relevant in the modern policy
environment with its strong focus on identifying and closing gaps. This
is because when interventions are given universally to all students,
Matthew Effects occur often, with disproportionate gains going to those
who had already been high performing (Ceci \& Papierno, 2005). The
question of what interventions should be provided -- and to whom --
becomes relevant. If the sole goal of a nation's policy is to close
gaps, targeted interventions that purposefully avoid providing the
intervention to higher performing groups is a path to consider. The
policy grows different if the goals vary (e.g., provide appropriate
challenge to students no matter their current performance level) or are
more multi-faceted (e.g., shrink gaps while also providing appropriate
challenge to all students).

Matthew Effects are particularly relevant for gifted education due to
pervasive excellence gaps. Excellence gaps represent performance
differences between sub-groups at the high end of achievement (J. A.
Plucker \& Peters, 2016). Typically demonstrated through gaps between
students of different race/ethnicity and income-level backgrounds in the
percent of students attaining high performance, excellence gaps are
large and pervasive. When achievement is unevenly distributed across
demographic groups, the existence of Matthew Effects suggest that
advanced educational opportunities like gifted programming will increase
each group's mean achievement while at the same times magnifying
discrepancies between groups.

The goal of this paper is to model the psychological underpinning of the
Matthew Effect as well as its limitations through the lens of Vygotsky's
Zone of Proximal Development (ZPD; Vygotsky (1980)). ZPD is the basic
psychological principle at the heart of this theoretical model. Although
the ZPD concept is typically attributed to Lev Vygotsky, the phenomenon
was actually first proposed by Dorthea McCarthy (1930). The ZPD
describes how learning emerges from the interaction between a student's
base of currently mastered skills and knowledge with a particular set of
learning experiences, and refers specifically to the set of skills that
a student cannot perform autonomously but can do with scaffolding from a
teacher or more competent peer. To further investigate the Matthew
Effect, we have produced a quantitative model for academic achievement
growth based on the ZPD (McBee, Makel, McCoach, nd) and developed an R
package for implementing these calculations (McBee, 2017). The package
is available for download from the first author's github site and
installed using the following R code:

\scriptsize

\begin{Shaded}
\begin{Highlighting}[]
\CommentTok{# install the function for loading R packages from github}
\KeywordTok{install.packages}\NormalTok{(}\StringTok{'devtools'}\NormalTok{)}
\KeywordTok{library}\NormalTok{(devtools)}
\CommentTok{# install the ZPDGrowthTrajectories package}
\KeywordTok{install_github}\NormalTok{(}\StringTok{"mcbeem/ZPDGrowthTrajectories"}\NormalTok{, }
               \DataTypeTok{build_vignettes=}\OtherTok{TRUE}\NormalTok{)}
\CommentTok{# load it}
\KeywordTok{library}\NormalTok{(ZPDGrowthTrajectories)}
\CommentTok{# view the package vignette}
\KeywordTok{vignette}\NormalTok{(}\StringTok{"Intro"}\NormalTok{, }\DataTypeTok{package=}\StringTok{"ZPDGrowthTrajectories"}\NormalTok{)}
\end{Highlighting}
\end{Shaded}

\normalsize

\section{Method}\label{method}

In this section, we provide a brief introduction to the features of the
quantitative ZPD model. The amount of learning during discrete time
interval \(t\) for individual \(i\) is given by

\small

\begin{equation}
\begin{split}
\textit{Learning}_{ti} = & (\textit{LearnRate}_i) \times \,  \\
  & \big[ ( \textit{dosage}_{ti})  \int_{0}^{\infty} \textit{ZPD}_{ti}(x) * S_t(x) \, dx  + \\ 
  & (1-\textit{dosage}_{ti}) (\textit{HomeEnv}_i)  \int_{0}^{\infty} \textit{ZPD}_{ti}(x) * H_t(x) \, dx \big]
\end{split}
\end{equation}

\normalsize

where * denotes the pointwise product, \(LearnRate_i\) the \(i\)th
student's learning rate, \(HomeEnv_i\) the \(i\)th student's home
environment, dosage a constant between 0 and 1 that describes the
relative exposure to school versus home during academic years, \(x\)
denotes achievement, \(ZPD_it(x)\) denotes the function describing the
\(i\)th student's zone of proximal development at time \(t\), \(S_t(x)\)
denotes the function describing the home curriculum at time \(t\), and
\(H_t(x)\) the function describing the home curriculum at time \(t\). As
previously described, we use the normalized normal density function as
our working model for \(ZPD(x)\), the normalized trapezoidal for
\(S(x)\), and the normalized beta for \(H(x)\). The first integral gives
the amount of learning due to school, the second integral the amount of
learning at home. The value of the second interval is multiplied by the
student's individual value of the home environment. Then these two
components are summed and multiplied by the student's learning rate to
yield the total growth (change in achievement) during interval \(t\).

Achievement is updated for time interval \(t\) by adding the amount of
learning during that interval to the student's prior achievement and
then multiplying the result by a value slightly smaller than 1.0 to
model decay due to forgetting.

\begin{equation}
\textit{Achievement}_{i,t}=(\textit{Achievement}_{i,t-1} + \textit{Learning}_{i,t})(1 - \textit{Decay}_i)
\end{equation}

\subsubsection{Academic achievement}\label{academic-achievement}

We define the term academic achievement to mean a student's location
along a sequenced, developmental curriculum within a specific domain.
Academic achievement therefore refers to the body of knowledge,
understanding, and skills that a student has mastered at a particular
moment in time. Academic growth or achievement growth is change
(typically positive) in achievement over time. Our numeric scale for
measuring achievement ranges from zero to positive infinity, where zero
is a meaningful quantity representing no knowledge or skills in the
domain. Specific values of achievement could refer to the location of
specific skill milestones or content mastery. For example, one could
(arbitrarily) set the mastery of mathematical operations expected of a
typical third grader to a value of 0.40 on the achievement metric. Prior
to plotting results, we rescaled the achievement values output by our
model to the NWEA MAP RIT metric to place them on a more intuitively
meaningful scale. The national norms from this test are displayed in
figure \ref{fig:MAPnorms}.

\begin{figure}[htbp]
\centering
\includegraphics{figures/MAPnorms-1.pdf}
\caption{\label{fig:MAPnorms}NWEA MAP national norms by subject and grade.}
\end{figure}

\subsubsection{Zone of Proximal
Development}\label{zone-of-proximal-development}

The ZPD is represented by a normal curve, normalized to a maximum value
of one, whose peak is slightly ahead of the student's current
achievement (figure \ref{fig:ZPD}). In the figure, a vertical line at
0.25 depicts the student's current achievement; her ZPD is the shaded
normal curve peaking at 0.27. The ZPD is not a single point but rather a
\enquote{spread} of values varying in intensity; its peak describes the
most efficacious level of instruction for stimulating academic growth.

\begin{figure}[htbp]
\centering
\includegraphics{figures/ZPD-1.pdf}
\caption{\label{fig:ZPD}Representation of the ZPD. The student's current
level of achievement in a particular subject is represented by the
vertical line. The student's ZPD is the shaded curve. Instruction is
effective at producing academic growth when it aligns with this range.}
\end{figure}

The ZPD is a function of the student's current achievement; as
achievement changes, the location of the ZPD moves along with it.

\subsubsection{School curriculum}\label{school-curriculum}

School curriculum is co-located on the same scale with academic
achievement in much the same way that IRT models place test scores and
item difficulties on a common scale. For example, one could imagine that
a typical Kindergarten reading curriculum ranges from 0.05 to 0.15,
where 0.05 might include basic letter and phoneme awareness and 0.15 a
limited ability to \enquote{sound out} simple words and knowledge of
roughly 200 sight words. Our model uses a normalized trapezoidal
distribution to represent the school curriculum due to the many shapes
it can assume with different choices of values for its parameters.
Figure \ref{fig:schoolCurr} illustrates how three students with varying
initial achievement, can experience the same school curriculum. The
student in the center panel would derive strong benefit from exposure to
this curriculum (manifesting as the most rapid growth) as it falls
within his ZPD. The curriculum is somewhat too difficult for the first
student and too easy for the third; neither of these students would
experience academic growth at the same pace due to this mismatch between
what they need and what they receive. Our implementation of the model in
our R package does allow for students to experience different curricula
(e.g., remedial, typical, or advanced) on an individual basis.

\begin{figure}[htbp]
\centering
\includegraphics{figures/schoolCurr-1.pdf}
\caption{\label{fig:schoolCurr}Interaction of the ZPD (blue) with the school
curriculum function (red). Panel 1: The student's current achievement is
substantially below the curriculum. The small area of the intersection
of these two curves indicates that this student will experience
relatively slow achievement growth during interval t. The instruction is
too advanced for optimal growth. Panel 2: The student experiences rapid
academic growth due to the correspondence between the ZPD and
instruction. Panel 3: The student experiences relatively slow growth
because the instruction is too basic.}
\end{figure}

\subsubsection{Home curriculum}\label{home-curriculum}

Children learn a great deal at home, particularly in early childhood and
in certain subjects such as reading. The rate of academic growth caused
by learning experiences at home is a function of the overlap between the
child's ZPD and what we call the \enquote{home curriculum}, which we
conceptualize as a downward-sloping function. Our R package implantation
represents the home curriculum as a normalized beta distribution with
the \(\alpha\) parameter fixed to 1. Figure \ref{fig:homeCurr} displays
the home curriculum function (with parameter \(\beta\) set to 5) with a
superimposed ZPD for a child with achievement = 0.20. The implication of
this downward slope is that academic growth due to home instruction is
most rapid in early childhood and becomes increasingly reliant on school
instruction in later grades.

\begin{figure}[htbp]
\centering
\includegraphics{figures/homeCurr-1.pdf}
\caption{\label{fig:homeCurr}Interaction of the ZPD (blue) with the home
curriculum function (red). The growth rate during interval t is
proportional to the intersection area of the ZPD curve (blue) with the
home curriculum function. As the student gains in achievement (moving
down the panels), the intensity of instruction from home is reduced.
This produces a diminishing achievement growth rate contribution from
home instruction.}
\end{figure}

\subsubsection{Individual difference
parameters}\label{individual-difference-parameters}

Our model incorporates four child-level individual difference
parameters: current achievement, learning rate, decay rate, and home
environment; these create variability in growth trajectories. The
learning rate represents individual differences in the pace of academic
growth for students with identical levels of ZPD engagement, and can be
considered to be a combination of general intelligence and the degree of
motivation, social and parental support. The decay rate is the rate at
which achievement will decay (due to forgetting) in the absence of
instruction. The home environment parameter describes how effectively
the home environment is at facilitating academic growth.

\subsection{Reproducing Empirical Results with the Theoretical
Model}\label{reproducing-empirical-results-with-the-theoretical-model}

We calibrated our model parameters against results presented by K. E.
Rambo-Hernandez and McCoach (2015), hereafter known as
\enquote{RHM2015}). In that paper, the authors fit a rich nonlinear
piecewise multilevel growth model to a large sample of MAP reading data.
The authors contrasted the academic year versus summer growth
trajectories for high (98th percentile) and average (16-84th percentile)
achieving groups, showing that the growth rate in reading for high
achieving students does not vary between the school year and the summer.
From our perspective, this paper offers a precise and highly constrained
target against which we could test the veracity of our model's synthetic
growth trajectories.

As a preliminary step for the analyses we present in this paper, we
identified parameters for our simulation that could produce data which
are highly similar to real data. To do this, we used our model to
generate trajectories from birth through the beginning of sixth grade
(matching the highest grade level reported in RHM2015). We took
cross-sectional samples from this simulated data at time points
corresponding with the grade levels in RHM2015; the beginning and end of
3rd, 4th, and 5th grades, and the beginning of 6th grade. We rescaled
the resulting achievement values to the MAP scale, added measurement
error, and fit a simplified version of the piecewise model employed by
RHM2015. Table \ref{tab:RHM2015} displays the empirical results from
RHM2015 versus results from fitting their model to our synthetic data.
The results are extraordinarily similar in both the fixed and random
components.

\begin{table}[tbp]
\begin{center}
\begin{threeparttable}
\caption{\label{tab:RHM2015}Comparison of piecewise growth model results from Rambo-Hernandez and McCoach (2015) to results from the same model fit to synthetic data generated from the theoretical model.}
\small{
\begin{tabular}{lrr}
\toprule
 & \multicolumn{2}{c}{Parameter (Std Err)} \\
\cmidrule(r){2-3}
1 & \multicolumn{1}{c}{2} & \multicolumn{1}{c}{3}\\
\midrule
Fixed effects &  & \\
\ \ \ Intercept & 193.72 (.084) & 190.29 (.053)\\
High-achieving & 25.24 (.244) & 25.79 (.313)\\
School year & 1.54 (.024) & 1.58 (.009)\\
School year * High-ach & -1.28 (.025) & -1.32 (.052)\\
Summer & -0.12 (.033) & -.083 (.015)\\
Summer * High-ach & 0.24 (.036) & 0.504 (.090)\\
School year squared & -0.02 (.001) & -.021 (.003)\\
School year squared * High-ach & 0.03 (.001) & .032 (.002)\\
Summer squared & .001 (.002) & -0.004 (.001)\\
\ \ \ Summer squared * High-ach & -0.001 (.002) & -.025 (.007)\\ \midrule
Random effects &  & \\
\ \ \ Level-1 eij & 29.890 & 28.896\\
\ \ \ Intercept & 30.601 & 30.761\\
\ \ \ School year growth & 0.100 & 0.286\\
\ \ \ Summer growth & - & -\\
\ \ \ School year growth squared & 0.000 & 0.000\\
\ \ \ Summer growth squared & - & -\\
\bottomrule
\addlinespace
\end{tabular}
}
\begin{tablenotes}[para]
\textit{Note.} 1: Model component. 2: Empirical result reported in Rambo-Hernandez \& McCoach (2015), Table 3. 3: Results from fitting the piecewise growth model from Rambo-Hernandez \& McCoach (2015) to synthetic data generated via the theoretical model.
\end{tablenotes}
\end{threeparttable}
\end{center}
\end{table}

We used the model parameters that can reproduce the RHM2015 results as
the basis of all the simulations reported in this paper.

\subsection{Origin of the Matthew
Effect}\label{origin-of-the-matthew-effect}

Figure \ref{fig:origin} illustrates the putative origin of the Matthew
effect. In this figure, prototypical academic growth trajectories for
three students are displayed beginning at birth. They grow at different
rates until they are observed at time \(t\), indicated by the vertical
reference line. It is observed that these three students differ in
achievement at time \(t\). If these students are observed again at a
later time, their differences in achievement will be larger than they
were when last observerd. This is due to the continuity of the process
leading to the difference in achievement at \(t\). The process that
caused the students' achievement to diverge at \(t\) will create
additional variance in achievement in later observations. Logically, the
highest-ranked student at time \(t\) must be the one who had experienced
the most rapid growth during the prior interval; absent some
discontinuous shock or achievement-limiting feature of the educational
environment, that student will continue to experience the fastest growth
in the subsequent interval.

\begin{figure}[htbp]
\centering
\includegraphics{figures/origin-1.pdf}
\caption{\label{fig:origin}Origin of the Matthew effect. X-axis: time.
Y-axis: achievement. Students are observed at a time point denoted by
the vertical dashed line. The process creating the discrepancies in
achievement at time t will continue unless interrupted or constrained,
resulting in growing inequality in achievement over time.}
\end{figure}

Of course, there is such an achievement-limiting feature that is highly
salient to school achievement: the limitation of the grade-level
curriculum to which the vast majority of students are exposed. For
example, Makel, Matthews, Peters, Rambo-Hernandez, and Plucker (2016)
recently found that 35\% of 5th graders score at least one grade level
ahead of placement in reading, with 24\% at least two grade levels above
placement. This raises the question of whether these students' school
experience will make any positive contribution at all to the development
of their reading skills. The results provided in RHM2015 provide an
answer; they observed no significant difference in the rate of summer
versus academic year growth in reading skills for high-achieving
students.

Thus, the excellence and achievement gaps that are the subject of the
Matthew effect are influenced by two countervailing forces: natural
variation in student background and learning rates create gaps that will
tend to grow over time, while curricular ceilings diminish gaps by
limiting the growth potential of the highest performing students. In the
following section, we use the quantified ZPD model to examine the
consequences of different types of interventions and examine their
effects on raw achievement as well as achievement and excellence gaps.

\section{Results}\label{results}

For each intervention considered, we used our theoretical model to
generate synthetic trajectories for three students. The trajectories
were generated from birth through the end of 5th grade; growing
trajectories beyond the fifth grade would have been excessively
speculative as we did not possess sufficiently detailed empirical data
for model calibration.

The three students differ in only three dimensions: learning rate, home
environment, and decay rate. The parameters are summarized in Table
\ref{tab:parameters}. We will refer to the difference between student
three and student two (e.g., the two highest performing students) as an
\emph{excellence gap}, and the difference between student three and
student one (e.g., the highest and lowest performing students) as an
\emph{achievement gap}.

\begin{table}[tbp]
\begin{center}
\begin{threeparttable}
\caption{\label{tab:parameters}Individual-differences parameters for the three simulated students}
\small{
\begin{tabular}{llll}
\toprule
Student & Learning rate & Home environment & Decay rate\\
\midrule
1 & 0.15 & 0.12 & 0.020\\
2 & 0.19 & 0.15 & 0.015\\
3 & 0.22 & 0.25 & 0.010\\
\bottomrule
\end{tabular}
}
\end{threeparttable}
\end{center}
\end{table}

\subsection{Model 1: No intervention / Standard multilevel
curriculum}\label{model-1-no-intervention-standard-multilevel-curriculum}

The results of model 1 are displayed in figure \ref{fig:model1}. This
model is based on the curriculum values that were used to reproduce the
RHM2015 results. In this model, students may be selected into either a
grade-level (typical) curriculum, a remedial curriculum, or an advanced
curriculum. The selection happens automatically; each student receives
whichever curriculum would produce the most growth in the time interval.
The upper panel of the figure shows the achievement trajectories for
each student expressed on the MAP reading metric from birth through the
end of fifth grade. The vertical dotted reference lines indicate the
time periods of the academic years. Before Kindergarten and during the
summers, growth occurs only because of exposure to home instruction;
during the school year growth results from the mixture of home and
school instruction.

In this condition, student three enters Kindergarten with substantially
mode advanced skills than the other students. This student experiences
little growth in grade K, grows very rapidly in grades 1 and 2, and
little growth thereafter. This is because the student is at the ceiling
of the grade level instruction by third grade and is no longer receiving
adequate opportunity to grow. This student's ZPD is mostly above what is
being taught even within the \enquote{advanced} curriculum.

Student one, on the other hand, enters Kindergarten with a relatively
deficient skillset. This student's growth levels off as well, but for a
different reason -- the curriculum is outpacing the student's growth
rate. After a few years, the student's ZPD calls increasingly below the
bulk of the curriculum, leading to slower growth.

Student two enters school at the \enquote{sweet spot} of developed
skill. This student's ZPD is consistently engaged with the school
curriculum, and as a result this student experiences rapid and
consistent growth. By the end of 5th grade, this student has nearly
caught up with student three in developed skill.

The lower panel of figure \ref{fig:model1} plots the excellence (student
3 vs 2) and achievement (student 3 vs 1) gaps over time. The excellence
gap slowly diminishes over time, while the achievement gap increases.

\begin{figure}[htbp]
\centering
\includegraphics{figures/model1-1.pdf}
\caption{\label{fig:model1}Synthetic trajectories and gaps for three
students. \emph{Condition}: no intervention; standard multilevel
curriculum (e.g., typical, remedial, and advanced). \emph{Panel 1}:
Achievement trajectories for birth through grade five in reading.
\emph{Panel 2:} Gaps (achievement differential). The
\enquote{achievement gap} is difference between student three and
student one; the \enquote{excellence gap} is the difference between
student three and student two.}
\end{figure}

The results from model 1, representing business-as-usual, will be used
as a baseline for the other models.

\subsection{Model 2: Intervention to increase learning
rates}\label{model-2-intervention-to-increase-learning-rates}

Model 2 examined the impact of a hypothetical intervention for
increasing learning rates. This condition approximates the effect of a
successful psychosocial intervention that improves motivation,
engagement, effort, or propensity to engage in deliberate practice. One
might even consider an increased learning rate in the standard
curriculum to be a desired secondary effect of academic enrichment. This
was modeled by adding a constant value of \(.05\) to each student's
learning rate (as presented in table \ref{tab:parameters}). No
alterations were made to the school curriculum. The results of this
model are displayed in figure \ref{fig:model2}. The results from model 1
are superimposed as faint dotted lines.

\begin{figure}[htbp]
\centering
\includegraphics{figures/model2-1.pdf}
\caption{\label{fig:model2}\emph{Condition}: intervention with additive
impact on learning rates; standard multilevel curriculum (e.g., typical,
remedial, and advanced). \emph{Panel 1}: Achievement trajectories.
\emph{Panel 2:} Gaps. Dotted reference lines display the no-intervention
results for comparison.}
\end{figure}

This intervention led to a reduction in both the excellence and the
achievement gaps. Increasing the learning rate of students three and two
did little to increase their grade-5 achievement; these students have
outpaced the curriculum by such an extent by third grade that their
growth almost ceases. The increased learning rate has enormous
consequences for student 1, who is now able to keep up with the
curriculum. As a result, student 1 has nearly caught up with students 2
and 3 by fifth grade.

\subsection{Model 3: Intervention to improve the educational quality of
the home
environment}\label{model-3-intervention-to-improve-the-educational-quality-of-the-home-environment}

In model 3, the learning rates were reset to their baseline values, but
each student's home environment was increased by a constant value of
\(.05\). This intervention is a prototypical abstraction of many which
seek to increase the academic contribution of the home environment. For
example, the Imagination Library sends one curated childrens' book per
month to all families with pre-Kindergartens. Other such programs
include summer camps, pre-K programs, educational apps and games, or
public library reading programs.

This model's results are displayed in figure \ref{fig:model3}. Due to
the downward-sloping nature of the home curriculum function (see figure
\ref{fig:homeCurr}), the benefits of this intervention are strongest in
early childhood and diminish with age. This intervention results in
students entering Kindergarten with more advanced skills than they would
otherwise have. By third grade, the benefits of this intervention have
been nearly eliminated for students 3 and 2 due to their encounter with
the curricular ceiling. Student 1 experiences some benefit, but is
ultimately still unable to keep up with the curriculum. This
intervention reduces the achievement gap while hardly affecting the
excellence gap.

\begin{figure}[htbp]
\centering
\includegraphics{figures/model3-1.pdf}
\caption{\label{fig:model3}\emph{Condition}: intervention with additive
impact on home environment; standard multilevel curriculum (e.g.,
typical, remedial, and advanced). \emph{Panel 1}: Achievement
trajectories. \emph{Panel 2:} Gaps. Dotted reference lines display the
no-intervention results for comparison.}
\end{figure}

\subsection{Model 4: Intervention to increase educational dosage (length
of school
day)}\label{model-4-intervention-to-increase-educational-dosage-length-of-school-day}

Model 4 displays the impact of an intervention which lengths the school
day, increasing the educational \enquote{dose.} The proportion of the
students' waking hours spend at school were increased from the 66\%
baseline to 80\%. Extending the length of the school day is the core
feature of interventions such as the Knowledge is Power Program (KIPP)
and Harlem Promise.

The results are displayed in figure \ref{fig:model4}. As can be seen in
the figure, the benefits of this type of intervention are
disproportionally experienced by the lowest-performing student, who
enjoys a substantial improvement in skill development over time. The
achievement gap by the beginning of 5th grade is nearly cut in half. The
intervention has almost no effect for the highest performing student,
who derives minimal benefit from prolonged exposure to instruction that
is too easy.

\begin{figure}[htbp]
\centering
\includegraphics{figures/model4-1.pdf}
\caption{\label{fig:model4}\emph{Condition}: intervention to increase
educational dose; standard multilevel curriculum (e.g., typical,
remedial, and advanced). \emph{Panel 1}: Achievement trajectories.
\emph{Panel 2:} Gaps. Dotted reference lines display the no-intervention
results for comparison.}
\end{figure}

\subsection{Model 5: Completely indivdualized
instruction}\label{model-5-completely-indivdualized-instruction}

Figure \ref{fig:model5} illustrates what might occur given completely
individualized instruction. In this model, the individual differences
parameters are set to their usual values (table \ref{tab:parameters}),
but the school curriculum function is altered to span the entire range
of achievement in each grade. As a result, the students' ZPDs are always
completely contained within the school curriculum. In this condition,
achievement growth occurs at the most rapid pace that the student can
sustain. Student 3's MAP reading score at the end of 5th grade is 299, a
value that is 4.30 standard deviations higher than the average
end-of-year 12th grader. Student 1's score of 224 is very close to the
mean reading score for 12th graders.

In this figure, all three students end fifth grade with much higher
levels of achievement than they would otherwise have reached, but the
achievement and excellence gaps are extreme.

\begin{figure}[htbp]
\centering
\includegraphics{figures/model5-1.pdf}
\caption{\label{fig:model5}\emph{Condition}: Individualized curriculum.}
\end{figure}

\subsection{Summary of achievement and gaps under the
models}\label{summary-of-achievement-and-gaps-under-the-models}

Tables \ref{tab:exc-gaps}, \ref{tab:ach-gaps}, and \ref{tab:ach-long}
summarize the raw MAP scores as well as the achievement and excellence
gaps under each model at the beginning of each grade.

\begin{table}[tbp]
\begin{center}
\begin{threeparttable}
\caption{\label{tab:ach-long}Achieved MAP score at the beginning of each grade by model and student.}
\small{
\begin{tabular}{lllll}
\toprule
Model & \multicolumn{1}{c}{Grade} & \multicolumn{1}{c}{1} & \multicolumn{1}{c}{2} & \multicolumn{1}{c}{3}\\
\midrule
1 & GradeK & 138.6 & 147.7 & 163.6\\
 & Grade1 & 153.5 & 161.7 & 170.4\\
 & Grade2 & 167.3 & 179.7 & 189.0\\
 & Grade3 & 179.0 & 195.0 & 206.8\\
 & Grade4 & 188.2 & 208.1 & 217.5\\
 & Grade5 & 195.2 & 217.7 & 224.4\\
 & Grade6 & 199.8 & 223.0 & 228.1\\ \midrule
2 & GradeK & 143.6 & 152.6 & 168.7\\
 & Grade1 & 159.8 & 165.6 & 173.9\\
 & Grade2 & 178.1 & 186.3 & 192.7\\
 & Grade3 & 193.3 & 205.0 & 211.3\\
 & Grade4 & 206.5 & 216.5 & 220.5\\
 & Grade5 & 216.2 & 223.5 & 226.7\\
 & Grade6 & 221.5 & 227.1 & 230.1\\ \midrule
3 & GradeK & 144.7 & 153.8 & 168.2\\
 & Grade1 & 157.9 & 164.7 & 173.1\\
 & Grade2 & 171.7 & 182.3 & 190.5\\
 & Grade3 & 183.5 & 197.8 & 207.9\\
 & Grade4 & 193.1 & 210.4 & 218.0\\
 & Grade5 & 200.5 & 219.1 & 224.6\\
 & Grade6 & 205.4 & 223.8 & 228.3\\ \midrule
4 & GradeK & 138.6 & 147.7 & 163.6\\
 & Grade1 & 155.6 & 163.1 & 170.8\\
 & Grade2 & 172.1 & 183.8 & 191.1\\
 & Grade3 & 186.1 & 202.1 & 210.3\\
 & Grade4 & 197.9 & 214.8 & 220.0\\
 & Grade5 & 207.2 & 222.5 & 226.3\\
 & Grade6 & 213.9 & 226.4 & 229.8\\ \midrule
5 & GradeK & 138.6 & 147.7 & 163.6\\
 & Grade1 & 155.8 & 169.5 & 189.0\\
 & Grade2 & 171.6 & 189.7 & 212.7\\
 & Grade3 & 186.2 & 208.6 & 235.3\\
 & Grade4 & 199.8 & 226.7 & 257.2\\
 & Grade5 & 212.7 & 243.9 & 278.4\\
 & Grade6 & 224.8 & 260.5 & 299.1\\
\bottomrule
\addlinespace
\end{tabular}
}
\begin{tablenotes}[para]
\textit{Note.} Model 1: No intervention, standard multilevel curriculum. Model 2: Improve learning rates. Model 3: Improve home environment. Model 4: Increase dose. Model 5: Individualized instruction.
\end{tablenotes}
\end{threeparttable}
\end{center}
\end{table}

\begin{table}[tbp]
\begin{center}
\begin{threeparttable}
\caption{\label{tab:exc-gaps}The excellence gap is defined as the achievement differential between the top two students. Excellence gaps at the beginning of each grade by model.}
\small{
\begin{tabular}{llllll}
\toprule
Grade & \multicolumn{1}{c}{Model1} & \multicolumn{1}{c}{Model2} & \multicolumn{1}{c}{Model3} & \multicolumn{1}{c}{Model4} & \multicolumn{1}{c}{Model5}\\
\midrule
GradeK & 15.9 & 16.1 & 14.4 & 15.9 & 15.9\\
Grade1 & 8.7 & 8.3 & 8.4 & 7.7 & 19.5\\
Grade2 & 9.3 & 6.5 & 8.1 & 7.4 & 23.0\\
Grade3 & 11.8 & 6.3 & 10.1 & 8.2 & 26.7\\
Grade4 & 9.4 & 4.0 & 7.6 & 5.2 & 30.5\\
Grade5 & 6.7 & 3.2 & 5.5 & 3.8 & 34.5\\
Grade6 & 5.2 & 3.0 & 4.5 & 3.4 & 38.6\\
\bottomrule
\addlinespace
\end{tabular}
}
\begin{tablenotes}[para]
\textit{Note.} Model 1: No intervention, standard multilevel curriculum. Model 2: Improve learning rates. Model 3: Improve home environment. Model 4: Increase dose. Model 5: Individualized instruction.
\end{tablenotes}
\end{threeparttable}
\end{center}
\end{table}

\begin{table}[tbp]
\begin{center}
\begin{threeparttable}
\caption{\label{tab:ach-gaps}Achievement gaps at the beginning of each grade by model.}
\small{
\begin{tabular}{llllll}
\toprule
Grade & \multicolumn{1}{c}{Model1} & \multicolumn{1}{c}{Model2} & \multicolumn{1}{c}{Model3} & \multicolumn{1}{c}{Model4} & \multicolumn{1}{c}{Model5}\\
\midrule
GradeK & 24.9 & 25.1 & 23.4 & 24.9 & 24.9\\
Grade1 & 16.9 & 14.0 & 15.2 & 15.2 & 33.2\\
Grade2 & 21.7 & 14.7 & 18.7 & 19.0 & 41.2\\
Grade3 & 27.8 & 18.0 & 24.4 & 24.2 & 49.2\\
Grade4 & 29.3 & 14.0 & 24.9 & 22.1 & 57.3\\
Grade5 & 29.2 & 10.5 & 24.2 & 19.1 & 65.7\\
Grade6 & 28.3 & 8.6 & 22.9 & 15.9 & 74.3\\
\bottomrule
\addlinespace
\end{tabular}
}
\begin{tablenotes}[para]
\textit{Note.} The achievement gap is defined as the achievement differential between the lowest- and highest-performing students. Model 1: No intervention, standard multilevel curriculum. Model 2: Improve learning rates. Model 3: Improve home environment. Model 4: Increase dose. Model 5: Individualized instruction.
\end{tablenotes}
\end{threeparttable}
\end{center}
\end{table}

\section{Discussion}\label{discussion}

This analysis examined the impact of four prototypical families of
educational interventions and their impact on achievement as well as
achievement disparities. The results suggest that Ceci and Papierno
(2005) contention that non-targeted intervention always magnifies gaps
is incomplete. Interventions that fail to raise the curricular ceiling
for high-performing students tend to shrink gaps, as their benefits
disproportionately accrue to the students who can benefit from the
grade-level curriculum.

According to this analysis, the only intervention which can meaningfully
increase achievement for the highest-performing students is some variant
of acceleration in which students are able to access advanced content
earlier in their school experience. When this artificial barrier to
continued growth is removed, students begin to accrue skills at rates
more commensurate with their potential learning speed -- differences
which create and increase the magnitude of gaps over time.

Larry Coleman was reportedly fond of saying that the most effective
educational system is the one that increases variance in student
achievement the most. The results of this analysis largely comport with
that sentiment. One can imagine academic environments which increase
achievement variance by denying low-performing students the opportunity
to grow. But the analysis reported in model 5 examined a
\enquote{perfect} educational system that is perfectly adaptive,
consistently delivering exactly the instruction that the student is
ready to receive. All of the students derived extraordinary achievement
benefit from this system. For example, under model 5, the lowest
performing student entered 6th grade with a MAP score equal to student 2
under the baseline model. This student's increase of 25 points (at the
end of 5th grade) is equivalent to 1.7 standard deviations with respect
to the MAP norms; a staggering increase by any standard. But the highest
performing student gains much more.

Effective gifted education interventions appear to magnify disparities
by their very nature. This analysis suggests that attempting to limit
disparities by pairing non-targeted interventions to increase learning
rate, home environment, educational dose would backfire. Once the
curricular ceiling is raised, the highest-achieving students would be
able to derive disproportionate benefits from those interventions as
well. The troubling issue is not the existence of achievement variance
but the distribution of academic growth trajectories across ethnic,
racial, and socioeconomic groups. Achievement variance is not the
concerning issue, but rather the covariance of achievement with
background characteristics that can lead to compounding
intergenerational disadvantage.

In addition to the policy implications, the quantitative theoretical
model we employed has significance as a novel approach to educational
science, broadly construed. It provides a structured framework for
analyzing educational policies, moving beyond the simple question of
basic effectiveness toward precise predictions regarding when, how, why,
and for whom interventions are anticipated to work. The model makes is
capable of making risky predictions that can be subjected to severe
tests against empirical data (see Mayo (1996), for a discussion of the
characteristics and benefits of severe tests). From both scientific and
pragmatic perspectives, a theory that is capable of generating point
predictions is far more useful that most qualitative educational
theories, from which only the vaguest predictions can be derived.
Furthermore, this approach moves beyond mere description of growth
trajectories to an investigation of their psychological origins,
providing a deeper integration of developmental psychology in
educational science.

From a practical perspective, our model provides educators with a
framework to consider how and when different students can benefit from
different educational experiences at different times. Additionally, it
helps connect gifted education with the larger education community
through universally relevant and shared constructs like growth, school
environment, home environment, and educational dose. Such alignment of
educational values offers the opportunity to facilitate dialog between
the gifted education and general education communities using shared
language and values. Conversations relying on universal terms, shared
values, as well as precise quantitative estimates will help all involved
participate in determining educational values and how and when
interventions should be implemented universally or to targeted groups at
targeted times.

\subsection{R packages used in this
work}\label{r-packages-used-in-this-work}

It is important that package authors receive citation credit for their
work. We used the following packages in this work: R (Version 3.4.0; R
Core Team, 2017) and the R-packages \emph{bindrcpp} (Version 0.2;
Müller, 2017), \emph{cowplot} (Version 0.9.2; Wilke, 2017a),
\emph{devtools} (Version 1.13.4; Wickham \& Chang, 2017), \emph{dplyr}
(Version 0.7.4; Wickham, Francois, Henry, \& Müller, 2017),
\emph{extrafont} (Version 0.17; Winston Chang, 2014), \emph{forcats}
(Version 0.3.0; Wickham, 2018), \emph{ggplot2} (Version 2.2.1; Wickham,
2009), \emph{ggridges} (Version 0.4.1; Wilke, 2017b), \emph{papaja}
(Version 0.1.0.9709; Aust \& Barth, 2018), \emph{reshape2} (Version
1.4.3; Wickham, 2007), \emph{trapezoid} (Version 2.0.0; Hetzel, 2012),
\emph{viridis} (Version 0.4.0; Garnier, 2017a, 2017b),
\emph{viridisLite} (Version 0.2.0; Garnier, 2017b), and
\emph{ZPDGrowthTrajectories} (Version 0.0.7; McBee, 2017).

\newpage

\section{References}\label{references}

\begingroup
\setlength{\parindent}{-0.5in} \setlength{\leftskip}{0.5in}

\hypertarget{refs}{}
\hypertarget{ref-R-papaja}{}
Aust, F., \& Barth, M. (2018). \emph{papaja: Create APA manuscripts with
R Markdown}. Retrieved from \url{https://github.com/crsh/papaja}

\hypertarget{ref-borkowski}{}
Borkowski, J. G., \& Peck, V. A. (1986). Causes and consequences of
metamemory in gifted children. \emph{Conceptions of Giftedness},
182--200.

\hypertarget{ref-ceci}{}
Ceci, J. S., \& Papierno, P. B. (2005). The rhetoric and reality of gap
closing: When the`` have-nots'' gain but the`` haves'' gain even more.
\emph{American Psychologist}, \emph{60}(2), 149.
doi:\href{https://doi.org/10.1037/0003-066X.60.2.149}{10.1037/0003-066X.60.2.149}

\hypertarget{ref-R-viridis}{}
Garnier, S. (2017a). \emph{Viridis: Default color maps from
'matplotlib'}. Retrieved from
\url{https://CRAN.R-project.org/package=viridis}

\hypertarget{ref-R-viridisLite}{}
Garnier, S. (2017b). \emph{ViridisLite: Default color maps from
'matplotlib' (lite version)}. Retrieved from
\url{https://CRAN.R-project.org/package=viridisLite}

\hypertarget{ref-R-trapezoid}{}
Hetzel, J. T. (2012). \emph{Trapezoid: The trapezoidal distribution}.
Retrieved from \url{https://CRAN.R-project.org/package=trapezoid}

\hypertarget{ref-huang}{}
Huang, F. L., Moon, T. R., \& Boren, R. (2014). Are the reading rich
getting richer? Testing for the presence of the matthew effect.
\emph{Reading and Writing Quarterly}, \emph{30}(2), 95--115.
doi:\href{https://doi.org/10.1080/10573569.2013.789784}{10.1080/10573569.2013.789784}

\hypertarget{ref-makel}{}
Makel, M., Matthews, M., Peters, S., Rambo-Hernandez, K., \& Plucker, J.
(2016). How can so many students be invisible? Large percentages of
american students perform above grade level. Baltimore, MD: Johns
Hopkins Institute for Education Policy. Retrieved from
\url{http://edpolicy.education.jhu.edu/wp-content/uploads/2016/08/StudentsinvisiblemastheadFINAL.pdf}

\hypertarget{ref-mayo}{}
Mayo, D. G. (1996). \emph{Error and the growth of experimental
knowledge}. Chicago: University of Chicago Press.

\hypertarget{ref-R-ZPDGrowthTrajectories}{}
McBee, M. (2017). \emph{ZPDGrowthTrajectories: Simulate achievement
growth trajectories using a theoretical model of academic achievement
based on the zone of proximal development}. Retrieved from
\url{https://github.com/mcbeem/ZPDGrowthTrajectories}

\hypertarget{ref-mccarthy}{}
McCarthy, D. A. (1930). \emph{The language development of the preschool
child.} Oxford, England: University of Minnesota Press.

\hypertarget{ref-merton}{}
Merton, R. K. (1968). The matthew effect in science: The reward and
communication systems of science are considered. \emph{Science},
\emph{159}(3810), 56--63.
doi:\href{https://doi.org/10.1126/science.159.3810.56}{10.1126/science.159.3810.56}

\hypertarget{ref-R-bindrcpp}{}
Müller, K. (2017). \emph{Bindrcpp: An 'rcpp' interface to active
bindings}. Retrieved from
\url{https://CRAN.R-project.org/package=bindrcpp}

\hypertarget{ref-plucker}{}
Plucker, J. A., \& Peters, S. J. (2016). \emph{Excellence gaps in
education: Expanding opportunities for talented students.} Cambridge,
MA.: Harvard Education Press.

\hypertarget{ref-protopapas}{}
Protopapas, A., Sideridis, G. D., Mouzaki, A., \& Simons, P. (2011).
Matthew effects in reading comprehension: Myth or reality? \emph{Journal
of Learning Disabilities}, \emph{44}(5), 402--420.
doi:\href{https://doi.org/10.1177/0022219411417568}{10.1177/0022219411417568}

\hypertarget{ref-R-base}{}
R Core Team. (2017). \emph{R: A language and environment for statistical
computing}. Vienna, Austria: R Foundation for Statistical Computing.
Retrieved from \url{https://www.R-project.org/}

\hypertarget{ref-rambo}{}
Rambo-Hernandez, K. E., \& McCoach, D. B. (2015). High-achieving and
average students? Reading growth: Contrasting school and summer
trajectories. \emph{The Journal of Educational Research}, \emph{108}(2),
112--129.
doi:\href{https://doi.org/10.1080/00220671.2013.850398}{10.1080/00220671.2013.850398}

\hypertarget{ref-vygotsky}{}
Vygotsky, L. S. (1980). \emph{Mind in society: The development of higher
psychological processes}. Cambridge, MA: Harvard University Press.

\hypertarget{ref-R-reshape2}{}
Wickham, H. (2007). Reshaping data with the reshape package.
\emph{Journal of Statistical Software}, \emph{21}(12), 1--20. Retrieved
from \url{http://www.jstatsoft.org/v21/i12/}

\hypertarget{ref-R-ggplot2}{}
Wickham, H. (2009). \emph{Ggplot2: Elegant graphics for data analysis}.
Springer-Verlag New York. Retrieved from \url{http://ggplot2.org}

\hypertarget{ref-R-forcats}{}
Wickham, H. (2018). \emph{Forcats: Tools for working with categorical
variables (factors)}. Retrieved from
\url{https://CRAN.R-project.org/package=forcats}

\hypertarget{ref-R-devtools}{}
Wickham, H., \& Chang, W. (2017). \emph{Devtools: Tools to make
developing r packages easier}. Retrieved from
\url{https://CRAN.R-project.org/package=devtools}

\hypertarget{ref-R-dplyr}{}
Wickham, H., Francois, R., Henry, L., \& Müller, K. (2017). \emph{Dplyr:
A grammar of data manipulation}. Retrieved from
\url{https://CRAN.R-project.org/package=dplyr}

\hypertarget{ref-R-cowplot}{}
Wilke, C. O. (2017a). \emph{Cowplot: Streamlined plot theme and plot
annotations for 'ggplot2'}. Retrieved from
\url{https://CRAN.R-project.org/package=cowplot}

\hypertarget{ref-R-ggridges}{}
Wilke, C. O. (2017b). \emph{Ggridges: Ridgeline plots in 'ggplot2'}.
Retrieved from \url{https://CRAN.R-project.org/package=ggridges}

\hypertarget{ref-R-extrafont}{}
Winston Chang. (2014). \emph{Extrafont: Tools for using fonts}.
Retrieved from \url{https://CRAN.R-project.org/package=extrafont}

\hypertarget{ref-zuckerman}{}
Zuckerman, H. (1965). Theory choice and problem choice in science.
\emph{Sociological Inquiry}, \emph{48}(3-4), 65--95.
doi:\href{https://doi.org/10.1111/j.1475-682X.1978.tb00819.x}{10.1111/j.1475-682X.1978.tb00819.x}

\endgroup





  \clearpage
  
  
  
\begin{appendix}
\section{}
R code for reproducing figure \ref{fig:model1}. Figures \ref{fig:model2}
through \ref{fig:model5} (as well as tables \ref{tab:ach-long},
\ref{tab:exc-gaps}, and \ref{tab:ach-gaps}) were produced using the same
code after making alterations to the \texttt{learning.rates} object
(model 2), \texttt{home.environments} object (model 3), \texttt{dosage=}
argument to the \texttt{plotGaps()} function (model 4), or the
\texttt{curriculum.starts} and \texttt{curriculum.widths} objects (model
5) as described in text.

\tiny

\begin{Shaded}
\begin{Highlighting}[]
\KeywordTok{library}\NormalTok{(devtools)}
\CommentTok{# download the ZPDGrowthTrajectories package from github; only needs to be done once}
\CommentTok{# uncomment the next line if you need to download the package}
\CommentTok{#install_github("mcbeem/ZPDGrowthTrajectories", build_vignettes=TRUE)}

\KeywordTok{library}\NormalTok{(ZPDGrowthTrajectories)}
\KeywordTok{library}\NormalTok{(ggplot2)}
\KeywordTok{library}\NormalTok{(cowplot)}
\KeywordTok{library}\NormalTok{(reshape2)}

\CommentTok{# set the individual differences parameters}
\NormalTok{learning.rates <-}\StringTok{ }\KeywordTok{c}\NormalTok{(.}\DecValTok{145}\NormalTok{, .}\DecValTok{175}\NormalTok{, .}\DecValTok{190}\NormalTok{) }\CommentTok{# learning rate }
\NormalTok{decay.rates <-}\StringTok{ }\KeywordTok{c}\NormalTok{(.}\DecValTok{01}\NormalTok{, .}\DecValTok{008}\NormalTok{, .}\DecValTok{001}\NormalTok{) }\CommentTok{# decay rate}
\NormalTok{home.environments <-}\StringTok{ }\KeywordTok{c}\NormalTok{(.}\DecValTok{09}\NormalTok{, .}\DecValTok{21}\NormalTok{, .}\DecValTok{34}\NormalTok{) }\CommentTok{# quality of home environment #.25}
\NormalTok{initial.achievements <-}\StringTok{ }\KeywordTok{c}\NormalTok{(}\DecValTok{0}\NormalTok{, }\DecValTok{0}\NormalTok{, }\DecValTok{0}\NormalTok{) }\CommentTok{# initial achievement all set to zero}

\CommentTok{# create the assignment object}
\NormalTok{assignment <-}\StringTok{ }\KeywordTok{c}\NormalTok{(}
\KeywordTok{rep}\NormalTok{(}\DecValTok{0}\NormalTok{, }\DecValTok{1825}\NormalTok{),   }\CommentTok{# time at home before school starts}
\KeywordTok{rep}\NormalTok{(}\KeywordTok{c}\NormalTok{(}\KeywordTok{rep}\NormalTok{(}\DecValTok{1}\NormalTok{,}\DecValTok{5}\NormalTok{), }\DecValTok{0}\NormalTok{, }\DecValTok{0}\NormalTok{), }\DecValTok{36}\NormalTok{),    }\CommentTok{# kindergarten}
\KeywordTok{rep}\NormalTok{(}\DecValTok{0}\NormalTok{, }\DecValTok{113}\NormalTok{),    }\CommentTok{# not in school days, including summers, holidays, and weekends}
\KeywordTok{rep}\NormalTok{(}\KeywordTok{c}\NormalTok{(}\KeywordTok{rep}\NormalTok{(}\DecValTok{2}\NormalTok{,}\DecValTok{5}\NormalTok{), }\DecValTok{0}\NormalTok{, }\DecValTok{0}\NormalTok{), }\DecValTok{36}\NormalTok{),    }\CommentTok{# first grade}
\KeywordTok{rep}\NormalTok{(}\DecValTok{0}\NormalTok{, }\DecValTok{113}\NormalTok{),    }\CommentTok{# not in school days, including summers, holidays, and weekends}
\KeywordTok{rep}\NormalTok{(}\KeywordTok{c}\NormalTok{(}\KeywordTok{rep}\NormalTok{(}\DecValTok{3}\NormalTok{,}\DecValTok{5}\NormalTok{), }\DecValTok{0}\NormalTok{, }\DecValTok{0}\NormalTok{), }\DecValTok{36}\NormalTok{),    }\CommentTok{# second grade}
\KeywordTok{rep}\NormalTok{(}\DecValTok{0}\NormalTok{, }\DecValTok{113}\NormalTok{),    }\CommentTok{# not in school days, including summers, holidays, and weekends}
\KeywordTok{rep}\NormalTok{(}\KeywordTok{c}\NormalTok{(}\KeywordTok{rep}\NormalTok{(}\DecValTok{4}\NormalTok{,}\DecValTok{5}\NormalTok{), }\DecValTok{0}\NormalTok{, }\DecValTok{0}\NormalTok{), }\DecValTok{36}\NormalTok{),    }\CommentTok{# third grade}
\KeywordTok{rep}\NormalTok{(}\DecValTok{0}\NormalTok{, }\DecValTok{113}\NormalTok{),    }\CommentTok{# not in school days, including summers, holidays, and weekends}
\KeywordTok{rep}\NormalTok{(}\KeywordTok{c}\NormalTok{(}\KeywordTok{rep}\NormalTok{(}\DecValTok{5}\NormalTok{,}\DecValTok{5}\NormalTok{), }\DecValTok{0}\NormalTok{, }\DecValTok{0}\NormalTok{), }\DecValTok{36}\NormalTok{),    }\CommentTok{# fourth grade}
\KeywordTok{rep}\NormalTok{(}\DecValTok{0}\NormalTok{, }\DecValTok{113}\NormalTok{),    }\CommentTok{# not in school days, including summers, holidays, and weekends}
\KeywordTok{rep}\NormalTok{(}\KeywordTok{c}\NormalTok{(}\KeywordTok{rep}\NormalTok{(}\DecValTok{6}\NormalTok{,}\DecValTok{5}\NormalTok{), }\DecValTok{0}\NormalTok{, }\DecValTok{0}\NormalTok{), }\DecValTok{36}\NormalTok{),    }\CommentTok{# fifth grade}
\KeywordTok{rep}\NormalTok{(}\DecValTok{0}\NormalTok{, }\DecValTok{113}\NormalTok{))    }\CommentTok{# not in school days, including summers, holidays, and weekends}

\NormalTok{curriculum.starts <-}\StringTok{ }\KeywordTok{structure}\NormalTok{(}\KeywordTok{c}\NormalTok{(}
\FloatTok{0.0715}\NormalTok{, }\FloatTok{0.2157}\NormalTok{, }\FloatTok{0.3187}\NormalTok{, }\FloatTok{0.3599}\NormalTok{, }\FloatTok{0.4011}\NormalTok{, }\FloatTok{0.432}\NormalTok{, }\CommentTok{# advanced}
\FloatTok{0.0406}\NormalTok{, }\FloatTok{0.1539}\NormalTok{, }\FloatTok{0.226}\NormalTok{, }\FloatTok{0.2981}\NormalTok{, }\FloatTok{0.3496}\NormalTok{, }\FloatTok{0.3908}\NormalTok{, }\CommentTok{# remedial}
\FloatTok{0.0715}\NormalTok{, }\FloatTok{0.1745}\NormalTok{, }\FloatTok{0.2569}\NormalTok{, }\FloatTok{0.329}\NormalTok{, }\FloatTok{0.3805}\NormalTok{, }\FloatTok{0.4217}\NormalTok{), }\CommentTok{# standard}
\DataTypeTok{.Dim =} \KeywordTok{c}\NormalTok{(6L, 3L)) }

\NormalTok{curriculum.widths <-}\StringTok{ }\KeywordTok{structure}\NormalTok{(}\KeywordTok{c}\NormalTok{(}
\FloatTok{0.125}\NormalTok{, }\FloatTok{0.104}\NormalTok{, }\FloatTok{0.083}\NormalTok{, }\FloatTok{0.062}\NormalTok{, }\FloatTok{0.041}\NormalTok{, }\FloatTok{0.02}\NormalTok{, }\CommentTok{# advanced}
\FloatTok{0.125}\NormalTok{, }\FloatTok{0.104}\NormalTok{, }\FloatTok{0.083}\NormalTok{, }\FloatTok{0.062}\NormalTok{, }\FloatTok{0.041}\NormalTok{, }\FloatTok{0.02}\NormalTok{, }\CommentTok{# remedial}
\FloatTok{0.125}\NormalTok{, }\FloatTok{0.104}\NormalTok{, }\FloatTok{0.083}\NormalTok{, }\FloatTok{0.062}\NormalTok{, }\FloatTok{0.041}\NormalTok{, }\FloatTok{0.02}\NormalTok{), }\CommentTok{# standard)}
\DataTypeTok{.Dim =} \KeywordTok{c}\NormalTok{(6L, 3L))}

\CommentTok{# modify the visualizeTrajectories function to add a legend and tweak the size and }
\CommentTok{# transparency of the lines}

\NormalTok{visualizeTrajectories2 <-}\StringTok{ }\NormalTok{function(trajectories) \{}
\NormalTok{if (}\KeywordTok{ncol}\NormalTok{(trajectories) !=}\StringTok{ }\DecValTok{3}\NormalTok{) \{}
\NormalTok{nstudents <-}\StringTok{ }\KeywordTok{nrow}\NormalTok{(trajectories)}
\NormalTok{days <-}\StringTok{ }\KeywordTok{ncol}\NormalTok{(trajectories) -}\StringTok{ }\DecValTok{1}
\NormalTok{trajectories <-}\StringTok{ }\NormalTok{reshape2::}\KeywordTok{melt}\NormalTok{(trajectories, }\DataTypeTok{measure.vars =} \KeywordTok{c}\NormalTok{(}\DecValTok{1}\NormalTok{:days +}\StringTok{ }\DecValTok{1}\NormalTok{))}
\NormalTok{trajectories[, }\DecValTok{2}\NormalTok{] <-}\StringTok{ }\KeywordTok{rep}\NormalTok{(}\KeywordTok{seq}\NormalTok{(}\DecValTok{1}\NormalTok{:days), }\DataTypeTok{each =} \NormalTok{nstudents)}
\KeywordTok{names}\NormalTok{(trajectories) <-}\StringTok{ }\KeywordTok{c}\NormalTok{(}\StringTok{"student"}\NormalTok{, }\StringTok{"day"}\NormalTok{, }\StringTok{"achievement"}\NormalTok{)}
\NormalTok{trajectories <-}\StringTok{ }\NormalTok{trajectories[}\KeywordTok{order}\NormalTok{(trajectories$student), ]}
\NormalTok{\}}
\NormalTok{p <-}\StringTok{ }\KeywordTok{ggplot}\NormalTok{(}\DataTypeTok{data =} \NormalTok{trajectories, }\KeywordTok{aes}\NormalTok{(}\DataTypeTok{x =} \NormalTok{day, }
\DataTypeTok{y =} \NormalTok{achievement, }\DataTypeTok{color =} \KeywordTok{factor}\NormalTok{(student))) +}\StringTok{ }\KeywordTok{geom_line}\NormalTok{(}
\DataTypeTok{size =} \FloatTok{0.6}\NormalTok{, }\DataTypeTok{alpha =} \FloatTok{0.6}\NormalTok{) +}\StringTok{ }\KeywordTok{geom_hline}\NormalTok{(}\DataTypeTok{yintercept =} \DecValTok{0}\NormalTok{, }
\DataTypeTok{col =} \StringTok{"gray"}\NormalTok{) +}\StringTok{ }\KeywordTok{geom_vline}\NormalTok{(}\DataTypeTok{xintercept =} \DecValTok{0}\NormalTok{, }\DataTypeTok{col =} \StringTok{"gray"}\NormalTok{) +}\StringTok{ }
\StringTok{        }\KeywordTok{theme}\NormalTok{(}\DataTypeTok{panel.background =} \KeywordTok{element_blank}\NormalTok{(), }
\DataTypeTok{panel.grid.major =} \KeywordTok{element_blank}\NormalTok{(), }\DataTypeTok{panel.grid.minor =} \KeywordTok{element_blank}\NormalTok{())}
\KeywordTok{return}\NormalTok{(p)}
\NormalTok{\}}

\CommentTok{# define function for plotting the trajectories and gaps}

\NormalTok{plotGaps <-}\StringTok{ }\NormalTok{function(assignment, learning.rates, decay.rates, initial.achievements, }
\NormalTok{home.environments, }\DataTypeTok{dosage=}\NormalTok{.}\DecValTok{66}\NormalTok{, curriculum.starts, curriculum.widths, }
\DataTypeTok{scale.intercept=}\FloatTok{118.5}\NormalTok{, }\DataTypeTok{scale.slope=}\DecValTok{245}\NormalTok{, }\DataTypeTok{ach.max=}\DecValTok{260}\NormalTok{, }\DataTypeTok{gap.max=}\DecValTok{60}\NormalTok{, }
\DataTypeTok{days.to.sample=}\DecValTok{1825}\NormalTok{+}\KeywordTok{seq}\NormalTok{(}\DecValTok{0}\NormalTok{, }\DataTypeTok{by=}\DecValTok{356}\NormalTok{, }\DataTypeTok{length.out=}\DecValTok{7}\NormalTok{)) \{}

\CommentTok{# generate the trajectories}

\NormalTok{trajectories <-}\StringTok{ }\KeywordTok{ZPDGrowthTrajectories}\NormalTok{(}\DataTypeTok{output.format=}\StringTok{"wide"}\NormalTok{,}
\DataTypeTok{days=}\DecValTok{4015}\NormalTok{,}
\DataTypeTok{assignment=}\NormalTok{assignment, }
\DataTypeTok{curriculum.start.points=}\NormalTok{curriculum.starts, }
\DataTypeTok{curriculum.widths=}\NormalTok{curriculum.widths,}
\DataTypeTok{dosage=}\NormalTok{dosage,}
\DataTypeTok{learning.rates=}\NormalTok{learning.rates,}
\DataTypeTok{decay.rates=}\NormalTok{decay.rates,}
\DataTypeTok{initial.achievements=}\NormalTok{initial.achievements,}
\DataTypeTok{home.environments=}\NormalTok{home.environments,}
\DataTypeTok{integration.points=}\DecValTok{200}\NormalTok{,}
\DataTypeTok{curriculum.lower.slope=}\DecValTok{8}\NormalTok{,}
\DataTypeTok{curriculum.upper.slope=}\DecValTok{10000}\NormalTok{,}
\DataTypeTok{alpha=}\DecValTok{1}\NormalTok{,}
\DataTypeTok{home.curriculum.shape1=}\DecValTok{1}\NormalTok{,}
\DataTypeTok{home.curriculum.shape2=}\DecValTok{5}\NormalTok{,}
\DataTypeTok{zpd.offset=}\NormalTok{.}\DecValTok{026}\NormalTok{,}
\DataTypeTok{zpd.sd=}\NormalTok{.}\DecValTok{039}\NormalTok{,}
\DataTypeTok{zpd.scale=}\NormalTok{.}\DecValTok{05}\NormalTok{,}
\DataTypeTok{decay.weight=}\NormalTok{.}\DecValTok{005}\NormalTok{,}
\DataTypeTok{useGPU=}\OtherTok{FALSE}\NormalTok{,}
\DataTypeTok{verbose=}\OtherTok{FALSE}\NormalTok{)}

\CommentTok{# rescale the achievement values to MAP scale}
\NormalTok{trajectories[,}\DecValTok{2}\NormalTok{:}\KeywordTok{ncol}\NormalTok{(trajectories)] <-}\StringTok{ }\KeywordTok{apply}\NormalTok{(trajectories[,}\DecValTok{2}\NormalTok{:}\KeywordTok{ncol}\NormalTok{(trajectories)], }
\KeywordTok{c}\NormalTok{(}\DecValTok{1}\NormalTok{,}\DecValTok{2}\NormalTok{), function(x) \{scale.intercept+(x*scale.slope)\})}

\CommentTok{# make the plot of trajectories (panel 1)}
\NormalTok{p1 <-}\StringTok{ }\KeywordTok{visualizeTrajectories2}\NormalTok{(trajectories)+}
\StringTok{    }\KeywordTok{xlim}\NormalTok{(}\DecValTok{0}\NormalTok{,}\DecValTok{3903}\NormalTok{)+}\KeywordTok{ylim}\NormalTok{(}\DecValTok{100}\NormalTok{, ach.max)+}\StringTok{  }\CommentTok{#1826}
\StringTok{    }\KeywordTok{labs}\NormalTok{(}\DataTypeTok{y=}\StringTok{"MAP score"}\NormalTok{, }\DataTypeTok{color=}\StringTok{"Student"}\NormalTok{)+}
\StringTok{    }\KeywordTok{geom_vline}\NormalTok{(}\DataTypeTok{xintercept=}\KeywordTok{seq}\NormalTok{(}\DecValTok{1826}\NormalTok{, }\DecValTok{3901}\NormalTok{, }\DataTypeTok{by=}\DecValTok{365}\NormalTok{), }\DataTypeTok{linetype=}\StringTok{"solid"}\NormalTok{, }\DataTypeTok{alpha=}\NormalTok{.}\DecValTok{2}\NormalTok{)+}
\StringTok{    }\KeywordTok{geom_vline}\NormalTok{(}\DataTypeTok{xintercept=}\KeywordTok{seq}\NormalTok{(}\DecValTok{1826+365-113}\NormalTok{, }\DecValTok{4100}\NormalTok{, }\DataTypeTok{by=}\DecValTok{365}\NormalTok{), }\DataTypeTok{linetype=}\StringTok{"solid"}\NormalTok{, }\DataTypeTok{alpha=}\NormalTok{.}\DecValTok{2}\NormalTok{)+}
\StringTok{    }\KeywordTok{annotate}\NormalTok{(}\StringTok{"text"}\NormalTok{, }\DataTypeTok{x =} \KeywordTok{seq}\NormalTok{(}\DecValTok{1826+110}\NormalTok{, }\DecValTok{4100}\NormalTok{, }\DataTypeTok{by=}\DecValTok{365}\NormalTok{), }\DataTypeTok{y =} \DecValTok{120}\NormalTok{, }
\DataTypeTok{label =} \KeywordTok{c}\NormalTok{(}\StringTok{" K "}\NormalTok{, }\StringTok{"1st"}\NormalTok{, }\StringTok{"2nd"}\NormalTok{, }\StringTok{"3rd"}\NormalTok{, }\StringTok{"4th"}\NormalTok{, }\StringTok{"5th"}\NormalTok{), }\DataTypeTok{size=}\FloatTok{3.2}\NormalTok{)+}
\StringTok{    }\KeywordTok{theme}\NormalTok{(}\DataTypeTok{legend.position=}\KeywordTok{c}\NormalTok{(.}\DecValTok{06}\NormalTok{, .}\DecValTok{75}\NormalTok{),}
\DataTypeTok{axis.text.x=}\KeywordTok{element_blank}\NormalTok{(),}
\DataTypeTok{axis.title.x=}\KeywordTok{element_blank}\NormalTok{(),}
\DataTypeTok{axis.ticks.x=}\KeywordTok{element_blank}\NormalTok{(),}
\DataTypeTok{axis.text.y=}\KeywordTok{element_text}\NormalTok{(}\DataTypeTok{size=}\DecValTok{11}\NormalTok{),}
\DataTypeTok{axis.title.y=}\KeywordTok{element_text}\NormalTok{(}\DataTypeTok{size=}\DecValTok{11}\NormalTok{),}
\DataTypeTok{legend.text=}\KeywordTok{element_text}\NormalTok{(}\DataTypeTok{size=}\DecValTok{11}\NormalTok{),}
\DataTypeTok{legend.title=}\KeywordTok{element_text}\NormalTok{(}\DataTypeTok{size=}\DecValTok{12}\NormalTok{))}

\CommentTok{# calculate the gaps by differencing student 3-2 and student 3-1}
\NormalTok{e_gap <-}\StringTok{ }\NormalTok{trajectories[}\DecValTok{3}\NormalTok{,}\DecValTok{2}\NormalTok{:}\KeywordTok{ncol}\NormalTok{(trajectories)] -}\StringTok{ }\NormalTok{trajectories[}\DecValTok{2}\NormalTok{,}\DecValTok{2}\NormalTok{:}\KeywordTok{ncol}\NormalTok{(trajectories)]}
\NormalTok{a_gap <-}\StringTok{ }\NormalTok{trajectories[}\DecValTok{3}\NormalTok{,}\DecValTok{2}\NormalTok{:}\KeywordTok{ncol}\NormalTok{(trajectories)] -}\StringTok{ }\NormalTok{trajectories[}\DecValTok{1}\NormalTok{,}\DecValTok{2}\NormalTok{:}\KeywordTok{ncol}\NormalTok{(trajectories)]}

\CommentTok{# bind into data frame. }
\CommentTok{# Gap type is varname "student" to enable visualizeTrajectories2 to be used}
\NormalTok{student <-}\StringTok{ }\KeywordTok{c}\NormalTok{(}\StringTok{"Excellence"}\NormalTok{, }\StringTok{"Achievement"}\NormalTok{)}
\NormalTok{gaps <-}\StringTok{ }\KeywordTok{cbind}\NormalTok{(student, }\KeywordTok{rbind}\NormalTok{(e_gap, a_gap))}

\CommentTok{# make the plot of gaps (panel 2)}
\NormalTok{p2 <-}\StringTok{ }\KeywordTok{visualizeTrajectories2}\NormalTok{(gaps)+}
\StringTok{    }\KeywordTok{xlim}\NormalTok{(}\DecValTok{0}\NormalTok{,}\DecValTok{3903}\NormalTok{)+}\KeywordTok{ylim}\NormalTok{(-}\DecValTok{10}\NormalTok{, gap.max)+}
\StringTok{    }\KeywordTok{labs}\NormalTok{(}\DataTypeTok{y=}\StringTok{"Gap"}\NormalTok{, }\DataTypeTok{color=}\StringTok{"Type"}\NormalTok{)+}
\StringTok{    }\KeywordTok{geom_vline}\NormalTok{(}\DataTypeTok{xintercept=}\KeywordTok{seq}\NormalTok{(}\DecValTok{1826}\NormalTok{, }\DecValTok{3901}\NormalTok{, }\DataTypeTok{by=}\DecValTok{365}\NormalTok{), }\DataTypeTok{linetype=}\StringTok{"solid"}\NormalTok{, }\DataTypeTok{alpha=}\NormalTok{.}\DecValTok{2}\NormalTok{)+}
\StringTok{    }\KeywordTok{geom_vline}\NormalTok{(}\DataTypeTok{xintercept=}\KeywordTok{seq}\NormalTok{(}\DecValTok{1826+365-113}\NormalTok{, }\DecValTok{4100}\NormalTok{, }\DataTypeTok{by=}\DecValTok{365}\NormalTok{), }\DataTypeTok{linetype=}\StringTok{"solid"}\NormalTok{, }\DataTypeTok{alpha=}\NormalTok{.}\DecValTok{2}\NormalTok{)+}
\StringTok{    }\KeywordTok{annotate}\NormalTok{(}\StringTok{"text"}\NormalTok{, }\DataTypeTok{x =} \KeywordTok{seq}\NormalTok{(}\DecValTok{1826+110}\NormalTok{, }\DecValTok{4100}\NormalTok{, }\DataTypeTok{by=}\DecValTok{365}\NormalTok{), }\DataTypeTok{y =} \NormalTok{-}\DecValTok{7}\NormalTok{, }
\DataTypeTok{label =} \KeywordTok{c}\NormalTok{(}\StringTok{" K "}\NormalTok{, }\StringTok{"1st"}\NormalTok{, }\StringTok{"2nd"}\NormalTok{, }\StringTok{"3rd"}\NormalTok{, }\StringTok{"4th"}\NormalTok{, }\StringTok{"5th"}\NormalTok{), }\DataTypeTok{size=}\FloatTok{3.2}\NormalTok{)+}
\StringTok{    }\CommentTok{# these color codes are ColorBrewer set1 colors 4 and 5}
\StringTok{    }\KeywordTok{scale_colour_manual}\NormalTok{(}\DataTypeTok{values =} \KeywordTok{c}\NormalTok{(}\StringTok{"#984ea3"}\NormalTok{, }\StringTok{"#ff7f00"}\NormalTok{))+}
\StringTok{    }\KeywordTok{theme}\NormalTok{(}\DataTypeTok{legend.position=}\KeywordTok{c}\NormalTok{(.}\DecValTok{06}\NormalTok{, .}\DecValTok{8}\NormalTok{),}
\DataTypeTok{axis.text.x=}\KeywordTok{element_blank}\NormalTok{(),}
\DataTypeTok{axis.title.x=}\KeywordTok{element_blank}\NormalTok{(),}
\DataTypeTok{axis.ticks.x=}\KeywordTok{element_blank}\NormalTok{(),}
\DataTypeTok{axis.text.y=}\KeywordTok{element_text}\NormalTok{(}\DataTypeTok{size=}\DecValTok{11}\NormalTok{),}
\DataTypeTok{axis.title.y=}\KeywordTok{element_text}\NormalTok{(}\DataTypeTok{size=}\DecValTok{11}\NormalTok{),}
\DataTypeTok{legend.text=}\KeywordTok{element_text}\NormalTok{(}\DataTypeTok{size=}\DecValTok{11}\NormalTok{),}
\DataTypeTok{legend.title=}\KeywordTok{element_text}\NormalTok{(}\DataTypeTok{size=}\DecValTok{12}\NormalTok{))}

\NormalTok{gap.summary <-}\StringTok{ }\NormalTok{gaps[,}\KeywordTok{c}\NormalTok{(}\DecValTok{1}\NormalTok{, days.to.sample}\DecValTok{+1}\NormalTok{)]}
\KeywordTok{names}\NormalTok{(gap.summary) <-}\StringTok{ }\KeywordTok{c}\NormalTok{(}\StringTok{"Type"}\NormalTok{, }\StringTok{"GradeK"}\NormalTok{, }\StringTok{"Grade1"}\NormalTok{, }\StringTok{"Grade2"}\NormalTok{, }\StringTok{"Grade3"}\NormalTok{, }
\StringTok{"Grade4"}\NormalTok{, }\StringTok{"Grade5"}\NormalTok{, }\StringTok{"Grade6"}\NormalTok{)}
\NormalTok{gap.summary <-}\StringTok{ }\KeywordTok{melt}\NormalTok{(gap.summary)}

\NormalTok{ach.summary <-}\StringTok{ }\NormalTok{trajectories[,}\KeywordTok{c}\NormalTok{(}\DecValTok{1}\NormalTok{, days.to.sample}\DecValTok{+1}\NormalTok{)]}
\KeywordTok{names}\NormalTok{(ach.summary) <-}\StringTok{ }\KeywordTok{c}\NormalTok{(}\StringTok{"Student"}\NormalTok{, }\StringTok{"GradeK"}\NormalTok{, }\StringTok{"Grade1"}\NormalTok{, }\StringTok{"Grade2"}\NormalTok{, }\StringTok{"Grade3"}\NormalTok{, }
\StringTok{"Grade4"}\NormalTok{, }\StringTok{"Grade5"}\NormalTok{, }\StringTok{"Grade6"}\NormalTok{)}
\NormalTok{ach.summary <-}\StringTok{ }\KeywordTok{melt}\NormalTok{(ach.summary, }\DataTypeTok{id.vars=}\StringTok{"Student"}\NormalTok{)}
\NormalTok{ach.summary <-}\StringTok{ }\NormalTok{ach.summary[}\KeywordTok{order}\NormalTok{(ach.summary$Student),]}

\NormalTok{nstudents <-}\StringTok{ }\KeywordTok{nrow}\NormalTok{(trajectories)}
\NormalTok{days <-}\StringTok{ }\KeywordTok{ncol}\NormalTok{(trajectories) -}\StringTok{ }\DecValTok{1}
\NormalTok{trajectories <-}\StringTok{ }\NormalTok{reshape2::}\KeywordTok{melt}\NormalTok{(trajectories, }\DataTypeTok{measure.vars =} \KeywordTok{c}\NormalTok{(}\DecValTok{1}\NormalTok{:days +}\StringTok{  }\DecValTok{1}\NormalTok{))}
\NormalTok{trajectories[, }\DecValTok{2}\NormalTok{] <-}\StringTok{ }\KeywordTok{rep}\NormalTok{(}\KeywordTok{seq}\NormalTok{(}\DecValTok{1}\NormalTok{:days), }\DataTypeTok{each =} \NormalTok{nstudents)}
\KeywordTok{names}\NormalTok{(trajectories) <-}\StringTok{ }\KeywordTok{c}\NormalTok{(}\StringTok{"student"}\NormalTok{, }\StringTok{"day"}\NormalTok{, }\StringTok{"achievement"}\NormalTok{)}
\NormalTok{trajectories <-}\StringTok{ }\NormalTok{trajectories[}\KeywordTok{order}\NormalTok{(trajectories$student), ]}

\NormalTok{nstudents <-}\StringTok{ }\KeywordTok{nrow}\NormalTok{(gaps)}
\NormalTok{days <-}\StringTok{ }\KeywordTok{ncol}\NormalTok{(gaps) -}\StringTok{ }\DecValTok{1}
\NormalTok{gaps <-}\StringTok{ }\NormalTok{reshape2::}\KeywordTok{melt}\NormalTok{(gaps, }\DataTypeTok{measure.vars =} \KeywordTok{c}\NormalTok{(}\DecValTok{1}\NormalTok{:days +}\StringTok{ }\DecValTok{1}\NormalTok{))}
\NormalTok{gaps[, }\DecValTok{2}\NormalTok{] <-}\StringTok{ }\KeywordTok{rep}\NormalTok{(}\KeywordTok{seq}\NormalTok{(}\DecValTok{1}\NormalTok{:days), }\DataTypeTok{each =} \NormalTok{nstudents)}
\KeywordTok{names}\NormalTok{(gaps) <-}\StringTok{ }\KeywordTok{c}\NormalTok{(}\StringTok{"student"}\NormalTok{, }\StringTok{"day"}\NormalTok{, }\StringTok{"achievement"}\NormalTok{)}
\NormalTok{gaps <-}\StringTok{ }\NormalTok{gaps[}\KeywordTok{order}\NormalTok{(gaps$student), ]}

\KeywordTok{return}\NormalTok{(}\KeywordTok{list}\NormalTok{(}\DataTypeTok{p1=}\NormalTok{p1, }\DataTypeTok{p2=}\NormalTok{p2, }\DataTypeTok{gap.summary=}\NormalTok{gap.summary, }\DataTypeTok{ach.summary=}\NormalTok{ach.summary, }
\DataTypeTok{trajectories=}\NormalTok{trajectories, }\DataTypeTok{gaps=}\NormalTok{gaps))}
\NormalTok{\}}

\NormalTok{model1 <-}\StringTok{ }\KeywordTok{plotGaps}\NormalTok{(}\DataTypeTok{assignment=}\NormalTok{assignment, }\DataTypeTok{learning.rates=}\NormalTok{learning.rates, }
\DataTypeTok{decay.rates=}\NormalTok{decay.rates, }\DataTypeTok{initial.achievements=}\NormalTok{initial.achievements,}
\DataTypeTok{home.environments=}\NormalTok{home.environments, }\DataTypeTok{dosage=}\NormalTok{.}\DecValTok{66}\NormalTok{,}
\DataTypeTok{curriculum.starts=}\NormalTok{curriculum.starts, }\DataTypeTok{curriculum.widths=}\NormalTok{curriculum.widths,}
\DataTypeTok{ach.max=}\DecValTok{240}\NormalTok{)}

\CommentTok{# show the plot}
\KeywordTok{plot_grid}\NormalTok{(model1$p1, model1$p2, }\DataTypeTok{ncol=}\DecValTok{1}\NormalTok{)}

\CommentTok{# save the data}
\NormalTok{gap1 <-}\StringTok{ }\NormalTok{model1$gap.summary}
\NormalTok{ach1 <-}\StringTok{ }\NormalTok{model1$ach.summary}

\CommentTok{# rename outcome column}
\KeywordTok{names}\NormalTok{(gap1)[}\DecValTok{3}\NormalTok{] <-}\StringTok{ "model1"}
\KeywordTok{names}\NormalTok{(ach1)[}\DecValTok{3}\NormalTok{] <-}\StringTok{ "model1"}
\end{Highlighting}
\end{Shaded}

\normalsize
\end{appendix}

\end{document}
